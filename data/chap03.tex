\chapter{绝热捷径与等温捷径构成的类卡诺循环热机}

\section{类卡诺循环热机模型}

\qquad 现在我们考虑一个由绝热捷径与等温捷径构成的类卡诺循环热机,它通过驱动一个依赖于时间的谐振子势$U_{0}(q, \lambda(t))= q^{2}/{\lambda(t)}^2 $对外做功。如图所示,这个循环由如下四个过程组成。

\begin{figure}[!htbp]
    \centering
    \def\svgwidth{0.6\columnwidth}
    \input{figures/p2.2.pdf_tex}
    \caption{热力学循环过程。点线对应于式\eqref{eq2.40},竖直虚线对应于等温。$0,1,2,3$代表循环中的四个状态,$A,B,C,D$代表四个过程。}
    \label{p3.1}
\end{figure}


\begin{center}
    {\bfseries A.等温膨胀}
\end{center}

在图\ref{p3.1}中,由实线连接的$0 \to 1$代表了由等温捷径实现的“等温膨胀”过程。正如在第\ref{cha2}章第\ref{sec2.3.2}节所阐释的,参数$\lambda$表征了系统的空间尺度,而该过程中参数$\lambda$变大了,故把过程A称为等温膨胀过程。

\section{类卡诺循环热机的功、熵、能量损失}

\section{与其他热机的比较}