\chapter{绝热捷径与等温捷径构成的类卡诺循环热机}
\section{绝热捷径与等温捷径中的能量学}
\R{李耿博士毕业论文,第一章随机能量学可作为参考}

考虑一个和温度为$T$的热浴接触的一维布朗粒子。它被施加一个依赖于时间的外势$U_0 (q,\lambda(t)) + U_1 (q,p,t)$,其中考虑到了在绝热捷径和等温捷径中我们施加的辅助势包含动量$U_1 (q,p,\lambda(t))$。粒子的哈密顿量为
\begin{equation}
    H=\frac{p^{2}}{2}+U_0 (q,\lambda(t)) + U_1 (q,p,t)
    \label{eq3.1}
\end{equation}
哈密顿量的全微分为
\begin{equation}
    \M{d} H=\left(\dot{p} p+\dot{q} \frac{\partial U_0}{\partial q} +  \dot{p} \PP{U_1}{p} + \dot{q} \PP{U_1}{q}  \right) \M{d} t+\left(\dot{\lambda} \frac{\partial U_0}{\partial \lambda} + \PP{U_1}{t} \right) \M{d} t
    \label{eq3.2}
\end{equation}
这提示我们定义沿着相轨迹$(q(t), p(t))$的能量差\cite{Tu2013}
\begin{equation}
     \Delta e \equiv H\left(t_{f}\right)-H\left(t_{i}\right),
     \label{eq3.3}
\end{equation}
输入功
\begin{equation}
    w \equiv \int_{t_{i}}^{t_{f}} \M{d} t \left( \dot{\lambda} \frac{\partial U_0}{\partial \lambda} + \PP{U_1}{t} \right)
    \label{eq3.4}
\end{equation}
这也与传统的随机热力学对轨道功的定义相符合。\cite{Sekimoto2010,Jarzynski1997,Sekimoto_1997}。还有吸收的热量
\begin{equation}
    q \equiv \int_{t_{i}}^{t_{f}} \M{d} t\left(\dot{p} p+\dot{q} \frac{\partial U_0}{\partial q} +  \dot{p} \PP{U_1}{p} + \dot{q} \PP{U_1}{q}  \right).
    \label{eq3.5}
\end{equation}
相轨连接了初时刻$t_i$对应的相点$(q_i , p_i )$和末时刻$t_f$对应的相点$(q_f , p_f )$。显然,对于每条相轨有
\begin{equation}
    \Delta e = w + q
    \label{eq3.6}
\end{equation}

而系统的分布函数$\rho (q,p,t)$由推广的福克-普朗克-克拉马斯方程\eqref{eq2.57}决定。在知道了$\rho (q,p,t)$之后就可以计算以上式子\eqref{eq3.4}-\eqref{eq3.6}的系综平均,这和文献\cite{Seifert2005,Shizume1995,Bizarro2011}的步骤是类似的。
\begin{equation}
    \Delta E \equiv\langle\Delta e\rangle=\left.\int \M{d} q \int \M{d} p(H \rho)\right|_{t_{i}} ^{t_{f}} ,
    \label{eq3.9}
\end{equation}
和
\begin{equation}
    W \equiv\langle w\rangle=\int_{t_{i}}^{t_{f}} d t \int d x \int \M{d} p\left( \dot{\lambda} \frac{\partial U_0}{\partial \lambda} + \PP{U_1}{t} \right).
    \label{eq3.10}
\end{equation}


\section{类卡诺循环热机模型}

\qquad 现在我们考虑一个由绝热捷径与等温捷径构成的类卡诺循环热机,它通过驱动一个依赖于时间的谐振子势$U_{0}(q, \lambda(t))= q^{2}/{\lambda(t)}^2 $对外做功。如图所示,这个循环由如下四个过程组成。

\begin{figure}[!htbp]
    \centering
    \def\svgwidth{0.6\columnwidth}
    \input{figures/p2.2.pdf_tex}
    \caption{热力学循环过程。点线对应于式\eqref{eq2.40},竖直虚线对应于等温。$0,1,2,3$代表循环中的四个状态,$A,B,C,D$代表四个过程。}
    \label{p3.1}
\end{figure}


\begin{center}
    {\bfseries A.等温膨胀}
\end{center}

在图\ref{p3.1}中,由实线连接的$0 \to 1$代表了由等温捷径实现的“等温膨胀”过程。正如在第\ref{cha2}章第\ref{sec2.3.2}节所阐释的,参数$\lambda$表征了系统的空间尺度,而该过程中参数$\lambda$变大了,故把过程A称为等温膨胀过程。

\section{类卡诺循环热机的功、熵、能量损失}

\section{与其他热机的比较}