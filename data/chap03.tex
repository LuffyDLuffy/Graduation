\chapter{绝热捷径与等温捷径构成的类卡诺循环热机}
\section{布朗粒子的能量学}
\qquad 在正式开始讨论我们构建的热机之前,为了方便讨论热机的效率、熵变等问题,让我们先花一点时间讨论布朗粒子的能量学。
\subsection{过阻尼布朗粒子的能量学}
\R{李耿博士毕业论文,第一章随机能量学可作为参考}

考虑一个和温度为$T$的热浴接触的一维布朗粒子。它被施加一个依赖于时间的外势$U(q,p,t) = U_0 (q,\lambda(t)) + U_1 (q,t)$,其中考虑到了在绝热捷径和等温捷径中我们施加的辅助势$U_1$。在过阻尼情况下,粒子的惯性的影响可以忽略,哈密顿量为
\begin{equation}
    H=U_0 (q,\lambda(t)) + U_1 (q,p,t)
    \label{eq31.1}
\end{equation}
哈密顿量的全微分为
\begin{equation}
    \M{d} H=\left(\dot{q} \frac{\partial U_0}{\partial q} + \dot{q} \PP{U_1}{q}  \right) \M{d} t+\left(\dot{\lambda} \frac{\partial U_0}{\partial \lambda} + \PP{U_1}{t} \right) \M{d} t
    \label{eq31.2}
\end{equation}
这提示我们定义沿着相轨迹$(q(t), p(t))$的能量差\cite{Tu2013}
\begin{equation}
     \Delta e \equiv H\left(t_{f}\right)-H\left(t_{i}\right),
     \label{eq31.3}
\end{equation}
输入功
\begin{equation}
    w \equiv \int_{t_{i}}^{t_{f}} \M{d} t \left( \dot{\lambda} \frac{\partial U_0}{\partial \lambda} + \PP{U_1}{t} \right)
    \label{eq31.4}
\end{equation}
这也与传统的随机热力学对轨道功的定义相符合。\cite{Sekimoto2010,Jarzynski1997,Sekimoto_1997}。还有吸收的热量
\begin{equation}
    q \equiv \int_{t_{i}}^{t_{f}} \M{d} t \dot{q} \frac{\partial U}{\partial q} .
    \label{eq31.5}
\end{equation}
相轨连接了初时刻$t_i$对应的相点$(q_i , p_i )$和末时刻$t_f$对应的相点$(q_f , p_f )$。显然,对于每条相轨有
\begin{equation}
    \Delta e = w + q
    \label{eq31.6}
\end{equation}

而系统的分布函数$\rho (q,p,t)$由的福克-普朗克-克拉马斯方程\eqref{eq2.45}决定,在知道了$\rho (q,p,t)$之后就可以计算以上式子\eqref{eq31.4}-\eqref{eq31.6}的系综平均,这和文献\cite{Seifert2005,Shizume1995,Bizarro2011}的步骤是类似的。得到能量差和输入功的系综平均为
\begin{equation}
    \Delta E \equiv\langle\Delta e\rangle=\left.\int \M{d} q (H \rho)\right|_{t_{i}} ^{t_{f}} ,
    \label{eq31.7}
\end{equation}
和
\begin{equation}
    W \equiv\langle w\rangle=\int_{t_{i}}^{t_{f}} \M{d} t \int \M{d} q \left[\rho \left( \dot{\lambda} \frac{\partial U_0}{\partial \lambda} + \PP{U_1}{t} \right) \right].
    \label{eq31.8}
\end{equation}

对于吸收的热的系综平均,由于存在$\dot{q}$,我们不好直接积分,需要做进一步的推导。系综平均为
\begin{equation}
    Q=\int_{t_i}^{t_f}\left\langle\dot{q} \frac{\partial U}{\partial q} \right\rangle \mathrm{d} t
    \label{eq31.9}
\end{equation}
现在分两步对上式\eqref{eq31.9}进行平均。

第一步,对系综中在$t$时刻经过$q$的轨道进行平均\R{(不懂至以下)},得到
\begin{equation}
    \langle\dot{x} \mid x, p, t\rangle=\frac{J}{\rho}
    \label{eq31.10}
\end{equation}
其中$J$是式\eqref{eq2.45.5}所定义的流。

第二步,利用分布函数$\rho(q,p,t)$对所有的$q$和$p$进行系综平均。于是,系统与热浴的热交换为
\begin{equation}
    \begin{split}
        Q&=\int_{t_i}^{t_f} \mathrm{d} t \int \mathrm{d} q \left(J \frac{\partial U}{\partial q}\right)\\
         &=\int_{t_i}^{t_f} \mathrm{d} t \int \mathrm{d} q \left[-\frac{1}{\gamma}  \frac{\partial U}{\partial q} \left(\frac{\partial U}{\partial q} \rho+\frac{1}{\beta} \frac{\partial \rho}{\partial q}\right) \right ]
    \end{split}   
    \label{eq31.11}
\end{equation}

% 现在我们将式\eqref{eq31.7},\eqref{eq31.8},\eqref{eq31.12}提到的计算能量差$\Delta E$、输入功$W$、吸收热$Q$的方法,应用于此模型中。对于过阻尼布朗粒子的等温捷径,注意到根据式\eqref{eq2.55},$U=U_0 + U_1 = q^2/{2 \lambda(t)^2}  -{\gamma \dot{\lambda}(t)  q^{2}}/{2 \lambda(t)}$,而根据\eqref{eq2.46}的假设,$\rho(q, t)= \mathrm{e}^{\beta\left[F(\lambda(t))-q^2/{2 \lambda(t)^2}\right]},\ F(\lambda) \equiv-\beta^{-1} \ln \left[\int_{-\infty}^{+\infty} e^{-\beta q^2/{2 \lambda(t)^2}} \M{d} q \right]$。不难得到,能量差

\subsection{欠阻尼布朗粒子的能量学}
\qquad 考虑一个和温度为$T$的热浴接触的一维布朗粒子。它被施加一个依赖于时间的外势$U(q,p,t) = U_0 (q,\lambda(t)) + U_1 (q,p,t)$,其中考虑到了在绝热捷径和等温捷径中我们施加的辅助势包含动量$p$。在欠阻尼情况下,粒子的惯性的影响不能忽略,哈密顿量为
\begin{equation}
    H=\frac{p^{2}}{2}+U_0 (q,\lambda(t)) + U_1 (q,p,t)
    \label{eq3.1}
\end{equation}
哈密顿量的全微分为
\begin{equation}
    \M{d} H=\left(\dot{p} p+\dot{q} \frac{\partial U_0}{\partial q} +  \dot{p} \PP{U_1}{p} + \dot{q} \PP{U_1}{q}  \right) \M{d} t+\left(\dot{\lambda} \frac{\partial U_0}{\partial \lambda} + \PP{U_1}{t} \right) \M{d} t
    \label{eq3.2}
\end{equation}
这提示我们定义沿着相轨迹$(q(t), p(t))$的能量差\cite{Tu2013}
\begin{equation}
     \Delta e \equiv H\left(t_{f}\right)-H\left(t_{i}\right),
     \label{eq3.3}
\end{equation}
输入功
\begin{equation}
    w \equiv \int_{t_{i}}^{t_{f}} \M{d} t \left( \dot{\lambda} \frac{\partial U_0}{\partial \lambda} + \PP{U_1}{t} \right)
    \label{eq3.4}
\end{equation}
这也与传统的随机热力学对轨道功的定义相符合。\cite{Sekimoto2010,Jarzynski1997,Sekimoto_1997}。还有吸收的热量
\begin{equation}
    q \equiv \int_{t_{i}}^{t_{f}} \M{d} t \left(\dot{p} p+\dot{q} \frac{\partial U}{\partial q} +  \dot{p} \PP{U}{p}\right).
    \label{eq3.5}
\end{equation}
相轨连接了初时刻$t_i$对应的相点$(q_i , p_i )$和末时刻$t_f$对应的相点$(q_f , p_f )$。显然,对于每条相轨有
\begin{equation}
    \Delta e = w + q
    \label{eq3.6}
\end{equation}

而系统的分布函数$\rho (q,p,t)$由推广的福克-普朗克-克拉马斯方程\eqref{eq2.57}决定。在知道了$\rho (q,p,t)$之后就可以计算以上式子\eqref{eq3.4}-\eqref{eq3.6}的系综平均,这和文献\cite{Seifert2005,Shizume1995,Bizarro2011}的步骤是类似的。得到能量差和输入功的系综平均为
\begin{equation}
    \Delta E \equiv \langle\Delta e\rangle=\left.\int \M{d} q \int \M{d} p(H \rho)\right|_{t_{i}} ^{t_{f}} ,
    \label{eq3.7}
\end{equation}
和
\begin{equation}
    W \equiv\langle w\rangle=\int_{t_{i}}^{t_{f}} \M{d} t \int \M{d} q \int \M{d} p \left[ \rho\left( \dot{\lambda} \frac{\partial U_0}{\partial \lambda} + \PP{U_1}{t} \right)\right] .
    \label{eq3.8}
\end{equation}

对于吸收的热的系综平均,由于存在$\dot{q}$和$\dot{p}$我们不好直接积分,需要做进一步的推导。系综平均为
\begin{equation}
    Q=\int_{t_i}^{t_f}\langle(\dot{x} \hat{\mathbf{x}}+\dot{p} \hat{\mathbf{p}}) \cdot \nabla H\rangle \mathrm{d} t
    \label{eq3.9}
\end{equation}
为了方便,上式\eqref{eq3.11}我们用了$(\dot{x} \hat{\mathbf{x}}+\dot{p} \hat{\mathbf{p}}) \cdot \nabla H  = \dot{p} p+\dot{q} \frac{\partial U}{\partial q} +  \dot{p} \PP{U}{p}$。
现在分两步对上式\eqref{eq3.9}进行平均。

第一步,对系综中在$t$时刻经过$x$且动量为$p$的轨道进行平均\R{(不懂至以下)},得到
\begin{equation}
    \langle\dot{x} \mid x, p, t\rangle=\frac{J_{x}}{\rho}, \quad\langle\dot{p} \mid x, p, t\rangle=\frac{J_{p}}{\rho}
    \label{eq3.10}
\end{equation}
其中$J_x$,$J_P$代表式\eqref{eq2.57}所定义的流矢量$J$的$q$分量和$p$分量。

第二步,利用分布函数$\rho(q,p,t)$对所有的$q$和$p$进行系综平均。于是,系统与热浴的热交换为
\begin{equation}
    Q=\int_{t_i}^{t_f} \mathrm{d} t \int \mathrm{d} q \int \mathrm{d} p(\bm{J} \cdot \nabla H)
    \label{eq3.11}
\end{equation}
再将流矢量$J$的定义代入上式\eqref{eq3.11},不难得到
\begin{equation}
    Q=-\int_{t_i}^{t_f} \mathrm{d} t \int \mathrm{d} q \int \mathrm{d} p\left[\gamma \rho\left(p+\frac{\partial U}{\partial p}\right)\left(p+\frac{\partial U}{\partial p}+\frac{1}{\beta \rho} \frac{\partial \rho}{\partial p}\right)\right]
    \label{eq3.12}
\end{equation}

\section{类卡诺循环热机模型}
\qquad 考虑一个由绝热捷径与等温捷径构成的类卡诺循环热机,它通过由一个依赖于时间的谐振子势$U_{0}(q, \lambda(t))= q^{2}/2{\lambda(t)}^2 $,来驱动布朗粒子对外做功。

在绝热捷径中,由于系统不与任何热浴接触,故没有热交换,$Q=0,\ \Delta E = W$。也没有过阻尼的情况,即惯性的影响不能忽略。由于在绝热捷径中,系统的初末状态依然是正则平衡态。初末的分布函数分别为式\eqref{eq2.32}和式\eqref{eq2.34},所以,我们依然可以用式\eqref{eq3.7}来计算能量差,于是
\begin{equation}
    \begin{split}
        W &= \Delta E\\
        &= \left.\int \M{d} q \int \M{d} p \left(\frac{1}{2} p^2 + \frac{1}{2 \lambda^2} q^2 \right) \rho \right|_{t_{i}} ^{t_{f}}\\
        &=\beta_{t_f}^{-1}-\beta_{t_i}^{-1}
    \end{split}
    \label{eq3.15.5}
\end{equation}

在等温捷径中,由于系统与热浴接触,可以分过阻尼和欠阻尼两种情况进行讨论。

\subsection{过阻尼布朗粒子构成的类卡诺循环热机}
\qquad 利用式\eqref{eq31.7},\quad \eqref{eq31.8},\quad \eqref{eq31.11}可以计算过阻尼布朗粒子的在等温捷径过程中的能量差$\Delta E$、输入功$W$、吸收热$Q$。注意到根据式\eqref{eq2.55},$U=U_0 + U_1 = q^2/{2 \lambda(t)^2}  -{\gamma \dot{\lambda}(t)  q^{2}}/{2 \lambda(t)}$,又根据\eqref{eq2.46}的假设,$\rho(q, t)= \mathrm{e}^{\beta\left[F(\lambda(t))-q^2/{2 \lambda(t)^2}\right]},\ F(\lambda) \equiv-\beta^{-1} \ln \left[\int_{-\infty}^{+\infty} \mathrm{e}^{-\beta q^2/{2 \lambda(t)^2}} \M{d} q \right]$。于是不难得到,
能量差\R{(似乎可以根据辅助势满足的微风方程直接计算,省去中间步骤,可以考虑)}
\begin{equation}
    \begin{split}
        \Delta E &=\left.\int \M{d} q \left[\rho\left(\frac{1}{2 \lambda^2}q^2  -\frac{\gamma \dot{\lambda}}{2 \lambda} q^{2}\right)\right]\right|_{t_{i}} ^{t_{f}}\\
         &=\left.\left(\frac{1}{2 \lambda^2}  -\frac{\gamma \dot{\lambda}}{2 \lambda}\right)\int \rho q^2 \M{d} q  \right|_{t_{i}} ^{t_{f}}\\
         &=\left.\left(\frac{1}{2 \lambda^2}  -\frac{\gamma \dot{\lambda}}{2 \lambda}\right)\sqrt{\frac{\beta}{2 \pi}} \frac{1}{\lambda} \int \mathrm{e}^{-\frac{\beta}{2 \lambda^2 } q^2} q^2 \M{d} q  \right|_{t_{i}} ^{t_{f}}\\
         &=\left.\left(\frac{1}{2 \lambda^2}  -\frac{\gamma \dot{\lambda}}{2 \lambda}\right)\sqrt{\frac{\beta}{2 \pi}} \frac{1}{\lambda} \frac{\sqrt{\pi}}{2}\left(\sqrt{\frac{\beta}{2}} \frac{1}{\lambda}\right)^{-3} \right|_{t_{i}}^{t_{f}}\\
         &=\left.\left(\frac{1}{2 \lambda^2}  -\frac{\gamma \dot{\lambda}}{2 \lambda}\right) \frac{\lambda^2}{\beta} \right|_{t_{i}}^{t_{f}}\\
         &=0
    \end{split}
    \label{eq3.13}
\end{equation}
其中最后一步用到了对$\lambda(t)$的限制条件\eqref{eq2.51}。注意到,和经典中理想气体的等温过程相似,粒子的能量没有变化。接下来计算输入功
\begin{equation}
    \begin{split}
        W &=\int_{t_{i}}^{t_{f}} \M{d} t \int \M{d} q \left[\rho \left(- \frac{\dot{\lambda}}{\lambda^3} q^2 -\frac{\gamma \ddot{\lambda}}{2 \lambda} q^2 +  \frac{\gamma \dot{\lambda}^2}{2 \lambda^2} q^2 \right) \right]\\
        &=\int_{t_{i}}^{t_{f}} \M{d} t  \left(- \frac{\dot{\lambda}}{\lambda^3} -\frac{\gamma \ddot{\lambda}}{2 \lambda} +  \frac{\gamma \dot{\lambda}^2}{2 \lambda^2} \right)\int \rho q^2 \M{d}   q\\
        &=\int_{t_{i}}^{t_{f}} \M{d} t  \left(- \frac{\dot{\lambda}}{\lambda^3} -\frac{\gamma \ddot{\lambda}}{2 \lambda} +  \frac{\gamma \dot{\lambda}^2}{2 \lambda^2} \right)\frac{\lambda^2}{\beta}\\
        &=\beta^{-1} \int_{t_{i}}^{t_{f}} \M{d} t  \left( -\frac{\dot{\lambda}}{\lambda} -\frac{\gamma \ddot{\lambda} \lambda }{2} +  \frac{\gamma \dot{\lambda}^2}{2} \right)
    \end{split}
    \label{eq3.14}
\end{equation}
再来看看吸收的热量
\begin{equation}
    \begin{split}
        Q &=\int_{t_i}^{t_f} \mathrm{d} t \int \mathrm{d} q \left[-\frac{1}{\gamma}  \left( \frac{1}{ \lambda^2}q  -\frac{\gamma \dot{\lambda}}{ \lambda} q \right) \left(\left( \frac{1}{ \lambda^2}q  -\frac{\gamma \dot{\lambda}}{ \lambda} q \right) \rho-\frac{1}{\lambda^2}q \rho \right) \right ]\\
        &=\int_{t_i}^{t_f} \mathrm{d} t \frac{1}{\gamma}  \left( \frac{1}{ \lambda^2}  -\frac{\gamma \dot{\lambda}}{ \lambda} \right) \frac{\gamma \dot{\lambda}}{ \lambda} \int \rho q^2 \mathrm{d} q\\
        &=\int_{t_i}^{t_f} \mathrm{d} t \frac{1}{\gamma}  \left( \frac{1}{ \lambda^2}  -\frac{\gamma \dot{\lambda}}{ \lambda} \right) \frac{\gamma \dot{\lambda}}{ \lambda} \frac{\lambda^2}{\beta}\\ 
        &=\beta^{-1} \int_{t_i}^{t_f} \mathrm{d} t \left( \frac{\dot{\lambda}}{ \lambda}  -\gamma \dot{\lambda}^2 \right)\\
        &=\beta^{-1} \ln{\frac{\lambda_f}{\lambda_i}} - \beta^{-1} \gamma \int_{t_i}^{t_f} \mathrm{d} t    \dot{\lambda}^2 
    \end{split}
    \label{eq3.15}
\end{equation}
不难验证有热平衡关系$\Delta E = W + Q$,接下来我们分析过阻尼布朗粒子构成的类卡诺热机的各个过程。

如图\ref{p3.1}所示,这个循环由如下四个过程组成。
\begin{figure}[!htbp]
    \centering
    \def\svgwidth{0.6\columnwidth}
    \input{figures/p3.1.pdf_tex}
    \caption{热力学循环过程。点线对应于式\eqref{eq2.40},竖直虚线对应于等温。0,\ 1,\ 2,\ 3代表循环中的四个状态,A,\ B,\ C,\ D代表四个过程。}
    \label{p3.1}
\end{figure}
点1, 2, 3, 4代表了系统的四个状态,这四个状态对应的参数如下表\ref{t3.1}所示。
\begin{table}[!htbp]
    \centering
    \caption{循环过程中各个状态所对应的参数}
    \begin{tabular}{cccc}
    \hline
    状态 & 时间$t$ & (等效)温度$\beta^{-1}$      & 势能参数$\lambda$ \\ 
    \hline
    0                   & $0(t_4)$       & $\beta_{\rm{h}}^{-1}$  & $\lambda_0$   \\ 

    1                   & $t_1$   & $\beta_{\rm{h}}^{-1}$ & $\lambda_1$   \\ 

    2                   & $t_2$   & $\beta_{\rm{c}}^{-1}$ & $\lambda_2$   \\ 

    3                   & $t_3$   & $\beta_{\rm{c}}^{-1}$ & $\lambda_3$   \\ 
    \hline
    \end{tabular}
    \label{t3.1}
\end{table}

\begin{center}
    {\bfseries A.等温膨胀}
\end{center}

在上图\ref{p3.1}中,由实线连接的$0 \to 1$代表了由等温捷径实现的“等温膨胀”过程。正如在第\ref{cha2}章第\ref{sec2.3.2}节所阐释的,参数$\lambda$表征了系统的空间尺度,而该过程中参数$\lambda$变大了,故把A过程称为等温膨胀过程。在A过程中,系统与高温热源$\beta_\mathrm{h}^{-1}$接触。可以由式\eqref{eq3.15}计算A过程系统吸收的能量,又鉴于在这个过程中由式\eqref{eq3.13}知有$\Delta E_{\rm{A}} = W_{\rm{A}} + Q_{\rm{A}}=0$,所以
\begin{equation}
    \Delta E_{\rm{A}}=0,\ -W_{\rm{A}}=Q_{\rm{A}} = \beta_{\rm{h}}^{-1} \ln{\frac{\lambda_1}{\lambda_0}} - \beta_{\rm{h}}^{-1} \gamma \int_{0}^{t_1} \mathrm{d} t    \dot{\lambda}^2 
    \label{eq3.16}
\end{equation}


\begin{center}
    {\bfseries B.绝热膨胀}
\end{center}

图\ref{p3.1}中由虚线连接的$1 \to 2$就是由绝热捷径所联系的“绝热膨胀”过程,膨胀一词也同样基于参数$\lambda$的增长。需要注意的是,在这个过程中我们只知道初末状态$1,\ 2$的等效温度$\beta_1 ,\ \beta_2$,中间过程的等效温度我们并不清楚,这也是我们用虚线作图的原因。好在,中间的过程的等效温度的知道与否并不影响我们接下来的讨论。

根据绝热捷径过程中的关系式\eqref{eq2.40},有$\lambda_2 =\lambda_1 \beta_{\rm{c}} / {\beta_{\rm{h}}}$。绝热捷径B过程中的能量差$\Delta E_{\rm{B}}$吸收热$Q_{\rm{B}}$,输入功$W_{\rm{B}}$正如本章开始所讨论的那样,
\begin{equation}
    W_{\rm{B}} = \Delta E_{\rm{B}} = (\beta_{\rm{c}}^{-1} - \beta_{\rm{h}}^{-1}),\  Q_{\rm{B}}=0
    \label{eq3.17}
\end{equation}

\begin{center}
    {\bfseries C.等温压缩}
\end{center}

在上图\ref{p3.1}中,由实线连接的$2 \to 3$代表了由等温捷径实现的“等温压缩”过程。在A过程中,系统与低温热源$\beta_\mathrm{c}^{-1}$接触。和A过程的讨论类似,不难得到
\begin{equation}
    \Delta E_{\rm{C}}=0,\ -W_{\rm{C}}=Q_{\rm{C}} = \beta_{\rm{c}}^{-1} \ln{\frac{\lambda_3}{\lambda_2}} - \beta_{\rm{c}}^{-1} \gamma \int_{t_2}^{t_3} \mathrm{d} t    \dot{\lambda}^2 
    \label{eq3.18}
\end{equation}

\begin{center}
    {\bfseries D.绝热压缩}
\end{center}

图\ref{p3.1}中由虚线连接的$3 \to 0$就是由绝热捷径所联系的“绝热压缩”过程。同样,根据\eqref{eq2.40},有$\lambda_3 =\lambda_0 \beta_{\rm{c}} / {\beta_{\rm{h}}}$。而且和B过程类似,不难得到
\begin{equation}
    W_{\rm{D}} = \Delta E_{\rm{D}} = (\beta_{\rm{h}}^{-1} - \beta_{\rm{c}}^{-1}),\  Q_{\rm{D}}=0
    \label{eq3.19}
\end{equation}
我们将各个过程的能量变化、输入功和吸收热总结如下表\ref{t3.2}

\begin{table}[!htbp]
    \caption{循环过程中的能量变化}
    \centering
    \begin{tabular}{c|c|c|c}
    \hline
    过程  & 能量差  & 输入功     & 吸收热 \\ 
    \hline
    A    & 0       & $-\beta_{\rm{h}}^{-1} \ln{\frac{\lambda_1}{\lambda_0}} - \beta_{\rm{h}}^{-1} \gamma \int_{0}^{t_1} \mathrm{d} t    \dot{\lambda}^2 $                    &$\beta_{\rm{h}}^{-1} \ln{\frac{\lambda_1}{\lambda_0}} - \beta_{\rm{h}}^{-1} \gamma \int_{0}^{t_1} \mathrm{d} t    \dot{\lambda}^2 $     \\ 
    \hline
    B    & $(\beta_{\rm{c}}^{-1} - \beta_{\rm{h}}^{-1})$ & $(\beta_{\rm{c}}^{-1} - \beta_{\rm{h}}^{-1})$ & 0   \\ 
    \hline
    C    & 0       & $-\beta_{\rm{c}}^{-1} \ln{\frac{\lambda_3}{\lambda_2}} - \beta_{\rm{c}}^{-1} \gamma \int_{t_2}^{t_3} \mathrm{d} t    \dot{\lambda}^2 $                    &$\beta_{\rm{c}}^{-1} \ln{\frac{\lambda_3}{\lambda_2}} - \beta_{\rm{c}}^{-1} \gamma \int_{t_2}^{t_3} \mathrm{d} t    \dot{\lambda}^2 $     \\ 
    \hline
    D    & $(\beta_{\rm{h}}^{-1} - \beta_{\rm{c}}^{-1})$ & $(\beta_{\rm{h}}^{-1} - \beta_{\rm{c}}^{-1})$                         & 0   \\ 
    \hline
    \end{tabular}
    \label{t3.2}
\end{table}

在分析完了整个循环的各个过程之后,来分析这个类卡诺循环热机的效率与功率。先看效率$\eta$,整个循环只有在A过程中从高温热源吸热$Q_{\rm{A}}$,而在C过程中发热$Q_{\rm{C}}$给低温热源。于是效率为
\begin{equation}
    \begin{split}
        \eta &= \frac{Q_{\rm{A}} + Q_{\rm{C}}}{Q_{\rm{A}}}\\ 
        &=1+ \frac{\beta_{\rm{h}}^{-1} \int_{0}^{t_1} \mathrm{d} t \left( {\dot{\lambda}}/{ \lambda}  -\gamma \dot{\lambda}^2 \right)}{\beta_{\rm{c}}^{-1} \int_{t_2}^{t_3} \mathrm{d} t \left( {\dot{\lambda}}/{ \lambda}  -\gamma \dot{\lambda}^2 \right)}
    \end{split}
    \label{3.20}
\end{equation}
而功率$P$为
\begin{equation}
    \begin{split}
        P&=-\frac{W}{t_4}\\
        &=\frac{Q_{\rm{A}} + Q_{\rm{C}}}{t_3}\\
        &=\frac{\beta_{\rm{h}}^{-1} \int_{0}^{t_1} \mathrm{d} t \left( {\dot{\lambda}}/{ \lambda}  -\gamma \dot{\lambda}^2 \right)+\beta_{\rm{c}}^{-1} \int_{t_2}^{t_3} \mathrm{d} t \left( {\dot{\lambda}}/{ \lambda}  -\gamma \dot{\lambda}^2 \right)}{t_4}
    \end{split}
    \label{eq3.21}
\end{equation}


\subsection{欠阻尼布朗粒子构成的类卡诺循环热机}
\qquad 同样地,利用式\eqref{eq31.7},\quad \eqref{eq31.8},\quad \eqref{eq31.11}可以计算欠阻尼布朗粒子的在等温捷径过程中的能量差$\Delta E$、输入功$W$、吸收热$Q$。注意到根据式\eqref{eq2.55},
\begin{equation}
    \begin{split}
        U&=U_0 + U_1 \\
    &= \frac{q^2}/{2 \lambda(t)^2}  - \frac{\dot{\lambda}(t)}{2 \gamma \lambda(t)}\left[(p-\gamma q)^{2}+ q^{2}/\lambda(t)^2\right]
    \end{split}  
    \label{eq3.22}
\end{equation}
又根据\eqref{eq2.58}的假设,
\begin{equation}
    \rho(q, t)= \mathrm{e}^{{\beta\left\{F-\frac{q^{2}}{2 \lambda^2}+\frac{\dot{\lambda}}{2 \gamma \lambda}\left[(p-\gamma q)^{2}+ \frac{q^{2}}{\lambda^2}\right]\right\}}}
\end{equation}
其中
\begin{equation}
    F(\lambda) \equiv-\beta^{-1} \ln  \int \M{d} q \int \M{d} p \mathrm{e}^{{\beta\left\{-\frac{q^{2}}{2 \lambda^2}+\frac{\dot{\lambda}}{2 \gamma \lambda}\left[(p-\gamma q)^{2}+ \frac{q^{2}}{\lambda^2}\right]\right\}}}  
    \label{eq3.23}
\end{equation}
。于是可以计算得到,能量差\R{(似乎可以根据辅助势满足的微风方程直接计算,省去中间步骤,可以考虑)}
\begin{equation}
    \begin{split}
        \Delta E &=\left.\int \M{d} q \left[\rho\left(\frac{1}{2 \lambda^2}q^2  -\frac{\gamma \dot{\lambda}}{2 \lambda} q^{2}\right)\right]\right|_{t_{i}} ^{t_{f}}\\
         &=\left.\left(\frac{1}{2 \lambda^2}  -\frac{\gamma \dot{\lambda}}{2 \lambda}\right)\int \rho q^2 \M{d} q  \right|_{t_{i}} ^{t_{f}}\\
         &=\left.\left(\frac{1}{2 \lambda^2}  -\frac{\gamma \dot{\lambda}}{2 \lambda}\right)\sqrt{\frac{\beta}{2 \pi}} \frac{1}{\lambda} \int \mathrm{e}^{-\frac{\beta}{2 \lambda^2 } q^2} q^2 \M{d} q  \right|_{t_{i}} ^{t_{f}}\\
         &=\left.\left(\frac{1}{2 \lambda^2}  -\frac{\gamma \dot{\lambda}}{2 \lambda}\right)\sqrt{\frac{\beta}{2 \pi}} \frac{1}{\lambda} \frac{\sqrt{\pi}}{2}\left(\sqrt{\frac{\beta}{2}} \frac{1}{\lambda}\right)^{-3} \right|_{t_{i}}^{t_{f}}\\
         &=\left.\left(\frac{1}{2 \lambda^2}  -\frac{\gamma \dot{\lambda}}{2 \lambda}\right) \frac{\lambda^2}{\beta} \right|_{t_{i}}^{t_{f}}\\
         &=0
    \end{split}
    \label{eq3.13}
\end{equation}
其中最后一步用到了对$\lambda(t)$的限制条件\eqref{eq2.51}。注意到,和经典中理想气体的等温过程相似,粒子的能量没有变化。接下来计算输入功
\begin{equation}
    \begin{split}
        W &=\int_{t_{i}}^{t_{f}} \M{d} t \int \M{d} q \left[\rho \left(- \frac{\dot{\lambda}}{\lambda^3} q^2 -\frac{\gamma \ddot{\lambda}}{2 \lambda} q^2 +  \frac{\gamma \dot{\lambda}^2}{2 \lambda^2} q^2 \right) \right]\\
        &=\int_{t_{i}}^{t_{f}} \M{d} t  \left(- \frac{\dot{\lambda}}{\lambda^3} -\frac{\gamma \ddot{\lambda}}{2 \lambda} +  \frac{\gamma \dot{\lambda}^2}{2 \lambda^2} \right)\int \rho q^2 \M{d}   q\\
        &=\int_{t_{i}}^{t_{f}} \M{d} t  \left(- \frac{\dot{\lambda}}{\lambda^3} -\frac{\gamma \ddot{\lambda}}{2 \lambda} +  \frac{\gamma \dot{\lambda}^2}{2 \lambda^2} \right)\frac{\lambda^2}{\beta}\\
        &=\beta^{-1} \int_{t_{i}}^{t_{f}} \M{d} t  \left( -\frac{\dot{\lambda}}{\lambda} -\frac{\gamma \ddot{\lambda} \lambda }{2} +  \frac{\gamma \dot{\lambda}^2}{2} \right)
    \end{split}
    \label{eq3.14}
\end{equation}
再来看看吸收的热量
\begin{equation}
    \begin{split}
        Q &=\int_{t_i}^{t_f} \mathrm{d} t \int \mathrm{d} q \left[-\frac{1}{\gamma}  \left( \frac{1}{ \lambda^2}q  -\frac{\gamma \dot{\lambda}}{ \lambda} q \right) \left(\left( \frac{1}{ \lambda^2}q  -\frac{\gamma \dot{\lambda}}{ \lambda} q \right) \rho-\frac{1}{\lambda^2}q \rho \right) \right ]\\
        &=\int_{t_i}^{t_f} \mathrm{d} t \frac{1}{\gamma}  \left( \frac{1}{ \lambda^2}  -\frac{\gamma \dot{\lambda}}{ \lambda} \right) \frac{\gamma \dot{\lambda}}{ \lambda} \int \rho q^2 \mathrm{d} q\\
        &=\int_{t_i}^{t_f} \mathrm{d} t \frac{1}{\gamma}  \left( \frac{1}{ \lambda^2}  -\frac{\gamma \dot{\lambda}}{ \lambda} \right) \frac{\gamma \dot{\lambda}}{ \lambda} \frac{\lambda^2}{\beta}\\ 
        &=\beta^{-1} \int_{t_i}^{t_f} \mathrm{d} t \left( \frac{\dot{\lambda}}{ \lambda}  -\gamma \dot{\lambda}^2 \right)\\
        &=\beta^{-1} \ln{\frac{\lambda_f}{\lambda_i}} - \beta^{-1} \gamma \int_{t_i}^{t_f} \mathrm{d} t    \dot{\lambda}^2 
    \end{split}
    \label{eq3.15}
\end{equation}
不难验证有热平衡关系$\Delta E = W + Q$,接下来我们分析过阻尼布朗粒子构成的类卡诺热机的各个过程。

如图\ref{p3.1}所示,这个循环由如下四个过程组成。
\begin{figure}[!htbp]
    \centering
    \def\svgwidth{0.6\columnwidth}
    \input{figures/p3.1.pdf_tex}
    \caption{热力学循环过程。点线对应于式\eqref{eq2.40},竖直虚线对应于等温。0,\ 1,\ 2,\ 3代表循环中的四个状态,A,\ B,\ C,\ D代表四个过程。}
    \label{p3.1}
\end{figure}
点1, 2, 3, 4代表了系统的四个状态,这四个状态对应的参数如下表\ref{t3.1}所示。
\begin{table}[!htbp]
    \centering
    \caption{循环过程中各个状态所对应的参数}
    \begin{tabular}{cccc}
    \hline
    状态 & 时间$t$ & (等效)温度$\beta^{-1}$      & 势能参数$\lambda$ \\ 
    \hline
    0                   & $0(t_4)$       & $\beta_{\rm{h}}^{-1}$  & $\lambda_0$   \\ 

    1                   & $t_1$   & $\beta_{\rm{h}}^{-1}$ & $\lambda_1$   \\ 

    2                   & $t_2$   & $\beta_{\rm{c}}^{-1}$ & $\lambda_2$   \\ 

    3                   & $t_3$   & $\beta_{\rm{c}}^{-1}$ & $\lambda_3$   \\ 
    \hline
    \end{tabular}
    \label{t3.1}
\end{table}

\begin{center}
    {\bfseries A.等温膨胀}
\end{center}

在上图\ref{p3.1}中,由实线连接的$0 \to 1$代表了由等温捷径实现的“等温膨胀”过程。正如在第\ref{cha2}章第\ref{sec2.3.2}节所阐释的,参数$\lambda$表征了系统的空间尺度,而该过程中参数$\lambda$变大了,故把A过程称为等温膨胀过程。在A过程中,系统与高温热源$\beta_\mathrm{h}^{-1}$接触。可以由式\eqref{eq3.15}计算A过程系统吸收的能量,又鉴于在这个过程中由式\eqref{eq3.13}知有$\Delta E_{\rm{A}} = W_{\rm{A}} + Q_{\rm{A}}=0$,所以
\begin{equation}
    \Delta E_{\rm{A}}=0,\ -W_{\rm{A}}=Q_{\rm{A}} = \beta_{\rm{h}}^{-1} \ln{\frac{\lambda_1}{\lambda_0}} - \beta_{\rm{h}}^{-1} \gamma \int_{0}^{t_1} \mathrm{d} t    \dot{\lambda}^2 
    \label{eq3.16}
\end{equation}


\begin{center}
    {\bfseries B.绝热膨胀}
\end{center}

图\ref{p3.1}中由虚线连接的$1 \to 2$就是由绝热捷径所联系的“绝热膨胀”过程,膨胀一词也同样基于参数$\lambda$的增长。需要注意的是,在这个过程中我们只知道初末状态$1,\ 2$的等效温度$\beta_1 ,\ \beta_2$,中间过程的等效温度我们并不清楚,这也是我们用虚线作图的原因。好在,中间的过程的等效温度的知道与否并不影响我们接下来的讨论。

根据绝热捷径过程中的关系式\eqref{eq2.40},有$\lambda_2 =\lambda_1 \beta_{\rm{c}} / {\beta_{\rm{h}}}$。绝热捷径B过程中的能量差$\Delta E_{\rm{B}}$吸收热$Q_{\rm{B}}$,输入功$W_{\rm{B}}$正如本章开始所讨论的那样,
\begin{equation}
    W_{\rm{B}} = \Delta E_{\rm{B}} = (\beta_{\rm{c}}^{-1} - \beta_{\rm{h}}^{-1}),\  Q_{\rm{B}}=0
    \label{eq3.17}
\end{equation}

\begin{center}
    {\bfseries C.等温压缩}
\end{center}

在上图\ref{p3.1}中,由实线连接的$2 \to 3$代表了由等温捷径实现的“等温压缩”过程。在A过程中,系统与低温热源$\beta_\mathrm{c}^{-1}$接触。和A过程的讨论类似,不难得到
\begin{equation}
    \Delta E_{\rm{C}}=0,\ -W_{\rm{C}}=Q_{\rm{C}} = \beta_{\rm{c}}^{-1} \ln{\frac{\lambda_3}{\lambda_2}} - \beta_{\rm{c}}^{-1} \gamma \int_{t_2}^{t_3} \mathrm{d} t    \dot{\lambda}^2 
    \label{eq3.18}
\end{equation}

\begin{center}
    {\bfseries D.绝热压缩}
\end{center}

图\ref{p3.1}中由虚线连接的$3 \to 0$就是由绝热捷径所联系的“绝热压缩”过程。同样,根据\eqref{eq2.40},有$\lambda_3 =\lambda_0 \beta_{\rm{c}} / {\beta_{\rm{h}}}$。而且和B过程类似,不难得到
\begin{equation}
    W_{\rm{D}} = \Delta E_{\rm{D}} = (\beta_{\rm{h}}^{-1} - \beta_{\rm{c}}^{-1}),\  Q_{\rm{D}}=0
    \label{eq3.19}
\end{equation}
我们将各个过程的能量变化、输入功和吸收热总结如下表\ref{t3.2}

\begin{table}[!htbp]
    \caption{循环过程中的能量变化}
    \centering
    \begin{tabular}{c|c|c|c}
    \hline
    过程  & 能量差  & 输入功     & 吸收热 \\ 
    \hline
    A    & 0       & $-\beta_{\rm{h}}^{-1} \ln{\frac{\lambda_1}{\lambda_0}} - \beta_{\rm{h}}^{-1} \gamma \int_{0}^{t_1} \mathrm{d} t    \dot{\lambda}^2 $                    &$\beta_{\rm{h}}^{-1} \ln{\frac{\lambda_1}{\lambda_0}} - \beta_{\rm{h}}^{-1} \gamma \int_{0}^{t_1} \mathrm{d} t    \dot{\lambda}^2 $     \\ 
    \hline
    B    & $(\beta_{\rm{c}}^{-1} - \beta_{\rm{h}}^{-1})$ & $(\beta_{\rm{c}}^{-1} - \beta_{\rm{h}}^{-1})$ & 0   \\ 
    \hline
    C    & 0       & $-\beta_{\rm{c}}^{-1} \ln{\frac{\lambda_3}{\lambda_2}} - \beta_{\rm{c}}^{-1} \gamma \int_{t_2}^{t_3} \mathrm{d} t    \dot{\lambda}^2 $                    &$\beta_{\rm{c}}^{-1} \ln{\frac{\lambda_3}{\lambda_2}} - \beta_{\rm{c}}^{-1} \gamma \int_{t_2}^{t_3} \mathrm{d} t    \dot{\lambda}^2 $     \\ 
    \hline
    D    & $(\beta_{\rm{h}}^{-1} - \beta_{\rm{c}}^{-1})$ & $(\beta_{\rm{h}}^{-1} - \beta_{\rm{c}}^{-1})$                         & 0   \\ 
    \hline
    \end{tabular}
    \label{t3.2}
\end{table}

在分析完了整个循环的各个过程之后,来分析这个类卡诺循环热机的效率与功率。先看效率$\eta$,整个循环只有在A过程中从高温热源吸热$Q_{\rm{A}}$,而在C过程中发热$Q_{\rm{C}}$给低温热源。于是效率为
\begin{equation}
    \begin{split}
        \eta &= \frac{Q_{\rm{A}} + Q_{\rm{C}}}{Q_{\rm{A}}}\\ 
        &=1+ \frac{\beta_{\rm{h}}^{-1} \int_{0}^{t_1} \mathrm{d} t \left( {\dot{\lambda}}/{ \lambda}  -\gamma \dot{\lambda}^2 \right)}{\beta_{\rm{c}}^{-1} \int_{t_2}^{t_3} \mathrm{d} t \left( {\dot{\lambda}}/{ \lambda}  -\gamma \dot{\lambda}^2 \right)}
    \end{split}
    \label{3.20}
\end{equation}
而功率$P$为
\begin{equation}
    \begin{split}
        P&=-\frac{W}{t_4}\\
        &=\frac{Q_{\rm{A}} + Q_{\rm{C}}}{t_3}\\
        &=\frac{\beta_{\rm{h}}^{-1} \int_{0}^{t_1} \mathrm{d} t \left( {\dot{\lambda}}/{ \lambda}  -\gamma \dot{\lambda}^2 \right)+\beta_{\rm{c}}^{-1} \int_{t_2}^{t_3} \mathrm{d} t \left( {\dot{\lambda}}/{ \lambda}  -\gamma \dot{\lambda}^2 \right)}{t_4}
    \end{split}
    \label{eq3.21}
\end{equation}


\section{类卡诺循环热机的功、熵、能量损失}

\section{与其他热机的比较}