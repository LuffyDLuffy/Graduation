\chapter{引言:热机的最大功率与效率问题}
热机的发明和使用对人类的生产生活产生了重大的影响,第一次工业革命和第二次工业革命都和热机的发展有紧密的关系. 一直以来,特别是近些年来,由于能源短缺的问题愈发突出,研究热机的效率问题吸引着一大批研究者.

早在1824,卡诺就指出(基于热质说)\cite{2005}:工作在相同高温热源和相同低温热源的所有热机中,以可逆热机的效率最高,而且这些可逆热机的效率都相同. 为$\eta _{\text{C}}=1-\frac{T_1}{T_2}$ 。这个问题似乎就这么解决了,然而必须注意的是可逆热机的要求相当于要求工作物质是准静态的. 也就是说,要想严格意义上达到卡诺热机的效率,热机一个循环的工作时间应当是无限长,这样的热机功率是0.

我们当然不可能接受功率为0的热机,所以得到热机保持在某一功率下(特别是最大功率)的最大效率成了摆在我们面前的问题,而这个问题促进了有限时间热力学的诞生. 

1975年,Curzon和Ahlborn研究了内可逆热机\cite{Curzon1975},该热机可在有限时间内完成一个循环. 利用Newton热运输规律,得到了该热机在最大功率下的效率为$\eta _{CA}=1-\sqrt{T_c/T_h}=1-\sqrt{1-\eta _C}$ ,其中$\eta _\text{C}$为卡诺效率。

2008年,T. Schmiedl 和 U. Seifert提出了布朗随机热机模型\cite{Schmiedl2008},该热机模型利用谐振子势场驱动布朗粒子做功,并得到了其在最大功率下的效率 . 

同年,涂展春推导出费曼棘轮热机\cite{Tu2008}在最大功率下的效率$\eta _{\text{F}}=\frac{\eta _{\text{C}}^{2}}{\eta _{\text{C}}-\left( 1-\eta _{\text{C}} \right) \ln \left( 1-\eta _{\text{C}} \right)}$. \cite{Tu2020}

……

同时涂展春\cite{Tu2008}在小温差条件下,即$\eta _{\text{C}}\rightarrow 0$. 将上述热机的效率按照 展开. 发现这些效率到二阶项都相同的,它们都普遍地满足$\eta _{\text{U}}=\frac{\eta _{\text{C}}}{2}+\frac{\eta _{\text{C}}^{2}}{8}+O\left( \eta _{\text{C}}^{3} \right)$. 这个规律在其他热机中也得到了验证,但也有一些热机不满足这个关系. 这引发了研究者对相关问题的思考,涂展春在紧耦合热机的范畴内对其中的原因进行了解释.\cite{Tu2020}

在这些形形色色的有限时间热机中,由Seifert等人提出的随机热机\cite{Schmiedl2008}占据一席之地. 随机热机通常是利用外势驱动与热源接触的布朗粒子做功. 其中的具体的实现过程可以有很大的不同,对应着不同的随机热机. 在这些不同的随机热机中, Campo等人利用绝热捷径构建的奥托热机\cite{DelCampo2014}引人注目. 而后,Deng等人发现绝热捷径能够提高奥托热机的效率,不论是经典的还是量子的.\cite{Deng2013} 涂展春在Schmiedl和Seifert工作\cite{Schmiedl2008}的基础上,考虑了他们在过阻尼情况下忽略的惯性的影响,构建了一种类卡洛热机\cite{Tu2013}. 惊讶地发现这种随机热机的效率等于Curzon和Ahlborn构建的内可逆热机\cite{Curzon1975}的效率 . 在涂展春构建的热机中\cite{Tu2013},布朗粒子在与时间相关的谐振子$U=\lambda ^2\left( t \right) x^2/2$的控制下,经历了类卡诺循环. 其中与卡诺循环中绝热过程对应的就是前面提到的绝热捷径,但与卡诺循环中等温过程对应的“等温过程”,却只是粒子与恒温热源接触,而非系统保持等温. 

这促使笔者思考是否有与卡诺循环中等温过程更加对应的过程,以实现类卡诺循环. 等温捷径\cite{Li2016}的提出为实现这个目标提供了契机,李耿等人给一个系统引入了辅助势,这个系统的演化本来是被一个含时哈密顿量所决定的,这个精心选择的辅助势可以使得系统在当前的哈密顿量下的状态,仿佛处于原哈密顿量的瞬时平衡态中一样,从而实现了“等温过程”,我们称之为等温捷径.

综上,笔者欲构造由绝热捷径和等温捷径构成的类卡诺循环,根据随机热力学和非平衡热力学中的典型方法,研究该热机工作过程中的功、熵、能量损失等参数,并考察它的效率与功率. 并与其他类型的热机,特别是涂展春构建的类卡洛热机\cite{Tu2013}进行比较,Campo等人构建的奥托热机\cite{DelCampo2014}. 考虑到Deng等人发现绝热捷径能够提高奥托热机的效率\cite{Deng2013},我们也期望等温捷径的引入能进一步提高热机的效率



\section{背景介绍:热机的功率与效率}
封面的例子请参看 \texttt{cover.tex}。主要符号表参看 \texttt{denation.tex},附录和
个人简历分别参看 \texttt{appendix01.tex} 和 \texttt{resume.tex}。里面的命令都很直
观,一看即会\footnote{你说还是看不懂?怎么会呢?}。

\section{字体命令}

